\section{Introduction}
The transportation sector is dominantly powered by petroleum-based
fuels~\cite{davis2009transportation}. In the United States, it is responsible
for 36\% of energy-related carbon dioxide emissions \cite{useia} and $\sim$50,000
premature deaths per year associated with particulate matter and ozone
emissions~\cite{caiazzo2013air}. Natural gas and hydrogen (H$_2$) are
alternative transportation fuels that emit fewer/less pollutants than
petroleum-based fuels, and thus their widespread adoption could mitigate
climate change~\cite{mcglade2015geographical} and improve air quality for human
health. Further, as the finite global petroleum resources are declining
rapidly~\cite{sorrell2010global}, the development of technologies for the
widespread adoption of sustainable transportation fuels, such as hydrogen, is
critical.

Natural gas, mostly methane, is considered a transition (to a renewable and
clean) fuel because it emits 25\% less carbon dioxide~\cite{eia2013much} and
fewer toxic byproducts~\cite{wang2000full} upon combustion per unit energy
produced compared to gasoline. From an economic standpoint, the supply of
natural gas in the United States is increasing as a result of hydraulic
fracturing and horizontal drilling techniques~\cite{usnatgassupply}. A positive
environmental outlook for natural gas, however, is predicated on mitigating
fugitive emissions (methane is itself a potent greenhouse
gas)~\cite{alvarez2012greater} and groundwater
contamination~\cite{osborn2011methane} from hydraulic fracturing.

Hydrogen (H$_2$) is the ultimate transportation fuel because it emits only
water when it electrochemically reacts with oxygen in a fuel cell to power a
vehicle. Currently, hydrogen (H$_2$) is primarily produced by steam reforming
of natural gas followed by the water-gas shift reaction, which emits carbon
dioxide \cite{crabtree2004hydrogen}. Notably, the environmental allure of
hydrogen is predicated on its production via a renewable means, e.g.\ splitting
water using wind-generated electricity.

% methane (0.036 MJ/L); hydrogen gas (0.01 MJ/L); liquid gasoline (34.2 MJ/L).
% (at 1 atm, liquid CH$_4$ at 111~K to achieve 22.2 MJ/L~\cite{makal2012methane}, liquid H$_2$ at 20 K to achieve 8~MJ/L \cite{suh2011hydrogen})
% (200 bar for CH$_4$ to achieve 9.2 MJ/L \cite{makal2012methane} and 700 bar for H$_2$ to achieve 4.5-5.3 MJ/L~\cite{unknown}).

At ambient conditions, both methane and hydrogen gas possess a very low
volumetric energy density compared to gasoline. Consequently, under storage
space constraints in passenger vehicles, natural gas and hydrogen must be
densified for onboard storage to achieve a reasonable driving range on a
``full'' tank of fuel. Traditional densification approaches are liquefaction,
at cryogenic temperatures and atmospheric pressure, or compression, at room
temperature and high pressures. Both approaches require expensive
infrastructure at refilling stations and significant energy input; e.g., the
energy input to liquify hydrogen is $\sim$30\% of its energy
content~\cite{bossel2003energy}. Moreover, high-pressure storage tanks are
heavy, thick-walled, and non-conformable, while cryogenic storage tanks are
bulky, expensive, and afflicted by boil-off losses~\cite{hasan2009minimizing}.

A promising approach to densify natural
gas~\cite{makal2012methane,mason2014evaluating} and
hydrogen~\cite{suh2011hydrogen,garcia2018benchmark} for vehicular storage at
room temperature is through physical adsorption in nanoporous
materials~\cite{schoedel2016role}. The internal surfaces of porous materials
attract gas molecules through van der Waals, electrostatic, etc.\ interactions
to achieve a higher adsorbed gas density than the bulk gas at the same
temperature and pressure, allowing for room temperature and lower-pressure
storage and alleviating many drawbacks of high-pressure, low-temperature
storage.

For a vehicle employing a porous material to store natural gas or hydrogen, the
(volumetric) \emph{deliverable capacity} of the gas is the primary
thermodynamic property that determines the driving
range~\cite{mason2014evaluating}. The adsorbed gas storage tank delivers the
gaseous fuel to the engine via an (assumed) isothermal pressure
swing~\cite{sircar2002pressure}. The deliverable capacity (see
Fig.~\ref{fig:delcap}) is the density of the gas in the material at the storage
pressure $\pfull$ minus the residual gas that remains adsorbed at the lowest
pressure $\pempty$ such that sufficient flow is maintained to feed the engine.
For commercial feasibility, the US Department of Energy (DOE) has set
deliverable capacity targets for both adsorbed methane and hydrogen storage and
delivery for vehicles. For methane, the Advanced Research Projects
Agency--Energy (ARPA-E) set a deliverable capacity target of 315\ L STP CH$_4$
per L of adsorbent at 298 K using a 65\ bar to 5.8\ bar pressure
swing~\cite{simon2015materials}. For hydrogen, the DOE set a series of
progressive targets at five year intervals, with the ultimate target of 0.05 kg
H$_2$/L~\cite{h2targetsDOE} using a 100\ bar to 5\ bar pressure swing at a
minimum of -40 $^\circ$C~\cite{allendorf2018assessment}. Thus far, despite the
emergence of highly tunable materials with large surface areas, such as
metal-organic frameworks~\cite{furukawa2013chemistry}, no porous material has
met these deliverable capacity targets~\cite{firlej2013understanding}.

% \begin{figure}
%     \centering
%     \subfloat[][]{
%         \includegraphics[width=0.6\columnwidth]{hkust-1_fancy.png} \label{fig:example_MOF}
%     }
%     \qquad
%     \subfloat[][]{
%         \includegraphics[width=0.8\columnwidth]{usable_capacity_illustration_toy_ish.pdf} \label{fig:delcap}
%     }
%     \caption{Gas storage and delivery using metal-organic frameworks (MOFs). (a) the crystal structure of an archetype MOF, CuBTC \cite{chui1999chemically}. (b) the methane adsorption isotherm in CuBTC \cite{chui1999chemically} at 298 K (blue) (data from Ref.~\cite{mason2014evaluating}). The deliverable capacity $\rho_D$ is illustrated as the density of gas in the MOF at the storage pressure $\pfull$ minus the density at the discharge pressure $\pempty$.
%     }
%     \label{fig:fig1}
% \end{figure}

\begin{figure}
    \centering
    \includegraphics[width=0.8\columnwidth]{Combine_Figure.pdf}
    \begin{minipage}{0.25\textwidth}
    \phantomsubcaption{\label{fig:example_MOF}}
    \end{minipage}
    \begin{minipage}{0.25\textwidth}
    \phantomsubcaption{\label{fig:delcap}}
    \end{minipage}
    
    \caption{\label{fig:fig1} Gas storage and delivery using metal-organic frameworks (MOFs). (\subref{fig:example_MOF}) the crystal structure of an archetype MOF, CuBTC \cite{chui1999chemically}. (\subref{fig:delcap}) the methane adsorption isotherm in CuBTC \cite{chui1999chemically} at 298 K (blue) (data from Ref.~\cite{mason2014evaluating}). The deliverable capacity $\rho_D$ is illustrated as the density of gas in the MOF at the storage pressure $\pfull$ minus the density at the discharge pressure $\pempty$.}
\end{figure}

To set realistic performance targets and optimally allocate research resources,
in this work, we present a theoretical framework that places an intrinsic upper
limit on the deliverable capacity of any gas in a rigid porous material and
uses as input the experimentally measured properties of the bulk gas. Our
extremum is provided by a substrate that offers a spatially uniform potential
energy field felt by the gas. Applying our framework to methane and hydrogen
gas, we find the US DOE deliverable capacity targets for natural gas and
hydrogen storage and delivery are theoretically possible, but sufficiently
close to the upper bound as to be impractical for any real, rigid porous
material. Optimistically, new paradigms outside the scope of applicability of
our theoretical framework, such as gas-induced structural transitions of the
material, hold promise for meeting these targets, as evidenced by flexible MOF
Co(bdp) which currently boasts the largest methane deliverable
capacity~\cite{mason2015methane}.

\section{Gas storage \& delivery by isothermal, pressure-swing adsorption}
Consider a pressure vessel onboard a vehicle (i.e. fuel tank) packed with
porous material. At the filling stage, the tank is connected to a (pure)
gaseous reservoir at pressure $\pfull$ and allowed to equilibrate. At this
point, the adsorbed gas tank is considered full. While driving, gas desorbs
from the adsorbent to the engine/fuel cell, driven by a pressure differential.
The tank is considered depleted/empty when the pressure has dropped to
$\pempty$, the pressure at which the flow rate of gas from the tank to the
engine is insufficient. However, given $\pempty \neq 0$ (pulling vacuum),
residual gas will remain trapped in the adsorbent. Therefore, the driving range
of the vehicle is primarily determined by the \emph{deliverable capacity} of
the gas in the material (see Fig.~\ref{fig:delcap}): the density of gas at
$\pfull$ minus that at $\pempty$. The isothermal, volumetric deliverable
capacity is an intrinsic property of the nanoporous material and its
interaction with the gas.

\section{Review of previous work}
There has been considerable work attempting to establish an upper bound on the
isothermal deliverable capacity in pressure-swing adsorption.

Early work showed that, in the simplified Langmuir model, there exists an
optimal free energy of adsorption (which determines the Langmuir constant) that
maximizes the deliverable capacity
$\rho_D$~\cite{matranga1992storage,bhatia2006optimum,simon2014optimizing}. If
the gas-substrate interaction is too weak (strong), too little (much) gas
adsorbs (is retained) at $\pfull$ ($\pempty$), diminishing $\rho_D$. An upper
bound on the deliverable capacity of a Langmuir material follows if each
adsorption site is endowed with the optimal free energy of adsorption. However,
remaining is the question of how many adsorption sites per volume a porous
material can practically offer, under the constraint that these adsorption
sites provide the optimal free energy of adsorption. Moreover, gas-gas
attractions, neglected in the Langmuir model, could recruit more gas in the
material at $\pfull$ than at $\pempty$ and enhance the deliverable
capacity~\cite{simon2014optimizing}.

G\'omez-Gualdr\'on \emph{et al.}~\cite{gomez2017impact} introduced a model that
accounted for gas-gas interactions via an intermolecular potential and
idealized the substrate in two different ways: (1) discrete adsorption sites
packed into an FCC lattice and (2) a volume endowed with a spatially uniform,
background potential energy field. According to molecular simulation of methane
adsorption, the ARPA-E deliverable capacity target of 315 L STP/L could be
reached in both of these idealized substrates. However, both models (i)
neglect the space occupied by atoms of the porous material that are needed to
endow the adsorption sites/volume with the attractive potential energy and (ii)
rely on a molecular model for methane.

A third body of work incorporated both gas-gas interactions \emph{and} steric
interactions of the gas with the atoms of the porous material. Simulations of
methane adsorption in hundreds of thousands of explicit nanoporous crystal
structures---both real and hypothetical---suggested that the ARPA-E deliverable
capacity target is infeasible~\cite{simon2015materials}; the highest simulated
methane deliverable capacity was 196 cm$^3$~STP/cm$^3$. Confidence in this
conclusion rests upon (i) the accuracy of the intermolecular potentials
describing the molecular interactions and (ii) the sufficient sampling of
material space, i.e.\ that the structures considered are representative of the
set of possible materials. To further explore material space and address
sensitivity to the intermolecular potentials: scaling the Lennard-Jones
potential well depths of material atoms to model enhanced
interactions~\cite{gomez2014exploring}, placing Lennard-Jones spheres in a unit
cell randomly to form ``pseudo-materials''~\cite{kaija2018high}, and generating
fictitious potential energy fields via a generative adversarial network trained
on zeolite structures~\cite{lee2019predicting} all generated model substrates
that failed to meet the ARPA-E methane deliverable capacity target.

% Brown and Freeman \cite{bhown2011analysis}

In this work, we place a rigorous upper bound on the deliverable capacity of a
pure gas in a rigid substrate by endowing a control volume with a spatially
uniform background energy field. Instead of using a molecular model for the
gas~\cite{gomez2017impact}, we use the experimental equation of state to
account for gas-gas interactions. In addition, we prove using the calculus of
variations that this spatially uniform substrate yields a maximal deliverable
capacity. We use our framework to place an upper bound on the deliverable
capacity of methane and hydrogen gas.

\section{An upper bound on the deliverable capacity of a pure gas in a rigid porous material}\label{sec:upper-bound}
\begin{figure}
  \centering
  \includegraphics[width=\columnwidth]{methane-298-rho-mu}
  \caption{The density of bulk methane, the ideal gas, and adsorbed gas in several porous materials at 298\ K as a function of chemical potential (bottom axis) and pressure (top axis). The bulk methane density is from the National Institute of Standards and Technology (NIST)~\cite{nist}. The methane adsorption isotherms in the porous materials are experimental data from Ref.~\cite{mason2014evaluating, furukawa2009storage}.
  }
  \label{fig:density-vs-mu-ch4}
\end{figure}

We now develop a thermodynamic framework that places an upper bound on the deliverable capacity of any pure gas in a rigid porous material.

The thermodynamic properties of a bulk, pure gas are characterized by an
equation of state. Of particular interest for gas storage and delivery is the
density of the gas $\rho_g(\mu,T)$ as a function of chemical potential $\mu$
and temperature $T$, is shown for methane gas in
Fig.~\ref{fig:density-vs-mu-ch4} at $T=298$\ K. For comparison, we also show
the density of methane adsorbed into several porous materials.

To place an upper bound on the deliverable capacity, consider a substrate whose
sole interaction with the gas is to introduce a spatially uniform potential
energy $\V$ for gas molecules in the control volume (where $\V<0$ for an
attractive potential). Because this interaction is uniform within the
substrate, the gas-gas interactions and thus fluid structure in such a
substrate are identical to those of the pure gas at the same density and
temperature. This allows us to obtain the adsorption properties of this model
substrate using the experimentally measured properties of the pure
gas~\cite{nist}. Intuitively, the deliverable capacity of a gas in such a
homogeneous substrate with the optimal potential energy is an upper bound
because, in a real material, (i) spatial inhomogeneity of the potential
energy results in some points in the control volume offering a suboptimal
attraction for the gas and (ii) the atoms required to create the potential
field exclude gas from occupying a fraction of the control volume.

The spatially uniform potential $\V$ representing gas-substrate interactions
behaves as an external potential and effectively shifts the chemical potential
(or, equivalently, molar Gibbs free energy) of the gas in the material, just as
gravitational potential energy causes the density of air to vary with altitude.
Consequently, the density of gas in our homogeneous substrate is:

\begin{align}
    \rho(\mu,T) &= \rho_g(\mu - \V,T). \label{eq:mof-density}
\end{align}
See Sec.~\ref{sec:V_shifts_chem_pot} for a derivation.

The deliverable capacity of gas in our homogeneous substrate is thus:

\begin{align}
    \rho_D(\V,T) &= \rho(\mufull,T) - \rho(\muempty,T),
    \label{eq:DofPhi}
    \\
    &= \rho_g(\mufull-\V,T) - \rho_g(\muempty-\V,T),
\end{align}
where $\mufull$ and $\muempty$ are the chemical potentials corresponding to the
pressures $\pfull$ and $\pempty$, respectively. Thus, $\rho_D$ for our
homogeneous substrate as a function of $\V$ is the difference between two
shifted versions of $\rho_g(\mu; T)$. Figure~\ref{fig:methane-298-D} shows the
two shifted bulk methane density curves and their difference. The horizontal
axis is the difference in molar Gibbs free energy of the gas, at fixed density (the density in the adsorbent) and temperature, between the gas within the substrate and the pure gas (\gst),
which is equal to $\V$ in our model, and is pressure-dependent in a real
crystal~(see Sec.~\ref{sec:phi-is-delta-g}). We see a potential $\V$ that
maximizes the deliverable capacity of our ideal substrate, balancing the need
to maximize the density at $\pfull$ against the need to minimize residual gas
retained at $\pempty$.

The optimal deliverable capacity of methane (298\ K, $\pfull=$ 65\ bar,
$\pempty=$ 5.8\ bar) in our ideal, spatially uniform substrate is 374\ L STP/L,
achieved for $\V_{opt} =$ 5.9\ kJ/mol.

An essential question is whether our idealized substrate, with spatially
uniform potential $\V_{opt}$ for the gas, places an upper bound on the
deliverable capacity in all rigid porous materials. In
Section~\ref{sec:proof-extremum}, we show that our idealized substrate with
spatially uniform potential $\V_{opt}$ yields an \emph{extremum} of the
deliverable capacity over all possible \emph{static potential energy fields},
provided the gas does not crystallize at the temperature and in the density
range of interest. The potential energy field is \emph{static} if it is
unaffected by the presence of the gas; consequently, our model does not apply
to flexible materials that undergo gas-induced conformation
changes~\cite{schneemann2014flexible}. We next argue that this extremum is a
maximum by addressing two alternative possibilities: the extremum could turn
out to be (i) a saddle point or (ii) a local, rather than global, maximum.

Fig.~\ref{fig:methane-298-D} clearly shows that $\V_{opt}$ provides a maximum
$\rho_D$ over all \emph{spatially uniform}, static potential energy fields. To
qualitatively argue that the spatially uniform potential $\V_{opt}$ provides a
maximum deliverable capacity over \emph{all} (including non-spatially uniform)
static potential energy fields (a broader claim), consider a
non-spatially-uniform variation on $\V_{opt}$. The effect of
this variation is to reduce the deliverable capacity: At low adsorbed gas
densities, the lowest-energy positions will be preferentially occupied, while
at higher densities, the additional gas molecules will be forced into higher
energy locations. Thus, the mean attraction will be lowest at \pfull\ and
highest at \pempty. As a result, $\gst$ increases monotonically with
pressure. As shown in Sec.~\ref{sec:monotonic}, this results in
the deliverable capacity of a non-uniform potential with a given $\gst$ at $\pfull$ or $\pempty$ being lower than the
deliverable capacity of a uniform potential with $\V=\gst$. Thus, a spatially uniform potential gives the maximal
deliverable capacity amongst all static potential energy fields that give the
same $\gst$, and the \emph{optimal} uniform potential $\V_{opt}$ will yield a
greater deliverable capacity over any static, non-uniform potential.

\begin{figure}
    \centering
    \includegraphics[width=0.95\columnwidth]{methane-298-n-vs-G}
    \caption{The deliverable capacity of methane as a function of the attractive Gibbs free energy $|\gst|$.
    Experimental deliverable capacities for several porous materials (data from Ref.~\cite{mason2014evaluating, furukawa2009storage}) are shown along with the experimental values for $\gst$ at the empty and full pressures shown as dots connected by a line.}
    \label{fig:methane-298-D}
\end{figure}

\section{Results}
Although our theoretical framework allows us to place an upper bound on the
deliverable capacity of many different gases in rigid porous materials, we
focus on the maximal deliverable capacity of methane and hydrogen gas in our
homogeneous substrate due to their application as transportation fuels. To
characterize the density of the gases, $\rho_g(p, T)$, we use experimental data
from NIST~\cite{nist}, which naturally includes quantum effects that
particularly affect hydrogen at low temperature~\cite{kumar2006quantum}. In the
context of storage onboard passenger vehicles, we compare our upper bound with
several prominent porous materials using experimental adsorption isotherms from
the literature; we also compare with deliverable capacity targets set by the US
DOE.

Figure~\ref{fig:methane-298-D} shows the upper bound for methane storage at
room temperature~(374~cm$^3$~STP/cm$^3$). In addition to the predicted maximum
deliverable capacity as a function of $\V$, the ARPA-E target of
315~cm$^3$~STP/cm$^3$~\cite{arpaemove} is shown for context. For an adsorbent
with at least 84\% void fraction, the ARPA-E target is theoretically possible.
The experimental deliverable capacities of several
adsorbents~\cite{mason2014evaluating} are also shown over a range of $\gst$
(converted from the measured adsorption isotherm as explained in
Sec.~\ref{sec:phi-is-delta-g}). For context, the highest observed
deliverable capacity for methane at room temperature in a rigid material is 208~cm$^3$~STP/cm$^3$~\cite{simon2015materials}.

\begin{figure}
    \centering
    \includegraphics[width=0.95\columnwidth]{hydrogen-298-n-vs-G}
    \caption{Deliverable capacity of hydrogen at room temperature as a function of the attractive Gibbs free energy $|\gst|$.  Experimental deliverable capacities for several porous materials (data from Ref.~\cite{mason2014evaluating, garcia2018benchmark}) are shown along with the experimental values for $\gst$ at the empty and full pressures shown as $+$'s connected by a line.}
    \label{fig:hydrogen-298-D}
\end{figure}

Storage of hydrogen is considerably more challenging owing to its relatively
weak interaction with adsorbents. For room temperature storage, the DOE
ULTIMATE deliverable capacity target~\cite{DOE} is within 6\% of the upper
bound. Figure~\ref{fig:hydrogen-298-D} shows the theoretical upper bound curve
for hydrogen storage along with experimental measurements for known adsorbents. The
DOE ULTIMATE deliverable capacity target is theoretically possible, however by
such a small margin that we can safely rule out the possibility of reaching
this target through storage and delivery of hydrogen in \emph{any} rigid
substrate at room temperature. Such a material would require a void fraction of
at least 94\%. On top of this, the DOE ULTIMATE target requires an optimal
$|\gst|$ of 10~kJ/mol which is far greater than what is found in observed
porous materials. This reflects the known fact that hydrogen interacts with
substrates far more weakly than methane does. One possibility that has been
pursued is storage at cryogenic temperatures. See
Section~\ref{sec:cryo-hydrogen} in the SI for the upper bound for hydrogen
storage at 77 K.

Why do we not experimentally observe materials approaching the upper bound on
the deliverable capacity? Foremost, any adsorbent substrate will be composed of
atoms, which will exclude the gas from some volume. Due to strong short-range
interactions, substrate atoms must be approximately uniformly distributed to achieve strong
attraction for gas atoms, imposing a limit on the pore
volume. In contrast, real materials have a spatially non-uniform attraction for
gas atoms. As a result, there are regions of space with
non-optimal attraction. For hydrogen, there is the further issue that there are
no known \emph{physical} interactions that are sufficiently strong to give an
optimal deliverable capacity at room temperature.

\section{Conclusion}
We have established an upper bound on the deliverable capacity via
pressure-swing adsorption in rigid porous solids based on experimentally
measured properties of pure gases. While these upper bounds do not rule out the
discovery of materials that reach current DOE targets for room temperature hydrogen and methane storage, they cast strong doubt
on the possibility of achieving these goals when we consider the additional
constraints imposed due to steric hindrance between substrate atoms and the
adsorbate. Our upper bound does indicate that those goals cannot be exceeded by
more than 16\% for methane and 6\% for hydrogen. Fortunately, there are some
limitations to these upper bounds which suggest avenues for future developments.

The first limitation of our proof is that we restricted ourselves to
\emph{isothermal} pressure swing storage (which is consistent with the targets set by DOE). By raising the temperature of the
adsorbent during gas discharge to drive off residual
gas~\cite{gomez2014exploring}, the deliverable capacity could be enhanced,
albeit at the cost of a more complicated engineering design of the fuel tank
and vehicle.

The second limitation of our proof arises in the assumption of a rigid
substrate. The rigid substrate acts as a static potential energy field for gas
molecules that is unchanged by the adsorption of gas. For most porous materials
this is a reasonable approximation, and this assumption is frequently made in
both the simulation and theory of porous materials~\cite{duren2009using}.
However, there are cases where the substrate can provide a very strong
gas-density-dependent interaction through structural
flexibility~\cite{schneemann2014flexible}. A flagship example is MOF
Co(bdp)~\cite{choi2008broadly}, which possesses a wine-rack-like topology
capable of hinge motion. At low methane pressure, Co(bdp) adopts a collapsed,
nonporous state, but expands to a porous state and fills with gas at higher
pressures~\cite{mason2015methane}. This allows Co(bdp) to fully expel its
residual gas at low pressures. Flexible materials could have significantly 
higher deliverable capacities, as our upper bound does not pertain to them.

We note that our upper bound can readily be applied to the storage of other
gasses of interest, with the proviso that the gas is far from crystallization.
Our code is available at \href{https://github.com/Jordan-Pommerenck/isothermal-gas-adsorption}{https://github.com/Jordan-Pommerenck/isothermal-gas-adsorption}.
